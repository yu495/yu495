\documentclass[platex,a4paper,12pt,dvipdfmx]{jsarticle}
\title{アカデミックスキル1最終レポート課題}
\author{s1300115 小島 優太}

\begin{document}
\maketitle

\newpage
\section{文章の内容を要約しなさい。(字数:300字以内)}

基礎科学の発見が社会の役に立つのは当然であり、一見役に立たないように思われた知識の追究や好奇心を応援することは、産業革命によって自然科学の知識や数学の方法が、技術の発展に目に見える形で役に立つようになったように、国益となる。

また、科学者が知的好奇心から行う自由な研究こそが、長い目で見て役立つ。というのも、基礎科学として価値のある発見は、幅広い自然現象を説明でき多くの科学の発展につながる。そうした大きな流れのすそ野には、社会に有益な技術への応用も当然含まれてくるので、つまり、基礎科学者が価値があると考える発見こそが、長い目で見て大きな役に立つのだ。
(276文字)

\section{(前略)あなたの考えを述べなさい。(字数:1000字~1200字厳守)}
私はこの考えは正しいと考える。理由は以下の3点である。

まず、能力のある研究者により多くの補助を与えようと考えることは、いたって自然なことであるからだ。学問の世界においてこれまでに達成したことがらによってその研究者が評価されるのは必然的なことであり、その功績によって研究者の良し悪しが決められるのは避けられないことである。なので、良い功績を残した研究者に多くの補助を与えるのは、理にかなっていて、当然のことといえるだろう。

次に、より優れた研究者のほうが、価値のある研究の方向を見定められる可能性が高いからだ。フェリックス・クラインの数学史書『一九世紀の数学』に「自分の楽しみのために考案されたこれら初期の知的遊戯がすべて、ずっと後になって意識するようになる大目標の布石だったのである。半ば遊びのような最初の力試しでさえ、深い意味を自覚しなくても、隠れている金脈にぴたりとつるはしを向けるというのはまさしく天才の予知能力に他ならない。」という表現があるように、能力のある研究者は初めから意識せずとも価値のある研究の方向に向かって進みだせる可能性が高く、そのような研究者により多くの支援をするというのは正しいといえる。

そして、能力のある研究者により多くの支援が行われるとなれば、より多くの研究者が能力があるとみなされようと研究に精を出すと考えたからだ。というのも、多くの研究者がより意欲的に研究を行うようになれば、それが基礎研究のさらなる発展に繋がり、さらには世界規模での技術の発展に繋がりうると考えるからだ。そうすれば、学問をより発展させる研究という行為が図らずともより価値のあるものになるだろう。

ただし、優れた功績を残せていない研究者に対しての支援も怠るべきではないと考える。なぜなら、「優れた功績を残せていない」というのはただ運が無かっただけであったりその研究にとても多くの時間を要するために成果を残せていないのであったりする可能性がある。なにより、少額の支援しか受けられないために満足に研究が行えないという可能性さえある。よって、一口に優れた功績を残せていないというだけで少ない支援しか受けられないというのも問題になり得ると考える。そこで、潜在的に能力のある研究者を見逃さないためにも、研究者とその研究を支援する人との距離をなるべく近くできるような環境が重要だといえる。

以上の点から、基礎研究への支援は研究者の探求心の良し悪しや研究者の能力に応じて行われるべきだというのは正しいと考える。
(1045文字)

\end{document}